\documentclass[letterpaper,11pt]{article}

\usepackage{latexsym}
\usepackage[empty]{fullpage}
\usepackage{titlesec}
\usepackage{marvosym}
\usepackage[usenames,dvipsnames]{color}
\usepackage{verbatim}
\usepackage{enumitem}
\usepackage[hidelinks]{hyperref}
\usepackage{fancyhdr}
\usepackage[english]{babel}
\usepackage{tabularx}
\input{glyphtounicode}


%----------FONT OPTIONS----------
% sans-serif
% \usepackage[sfdefault]{FiraSans}
% \usepackage[sfdefault]{roboto}
% \usepackage[sfdefault]{noto-sans}
% \usepackage[default]{sourcesanspro}

% serif
% \usepackage{CormorantGaramond}
% \usepackage{charter}


\pagestyle{fancy}
\fancyhf{} % clear all header and footer fields
\fancyfoot{}
\renewcommand{\headrulewidth}{0pt}
\renewcommand{\footrulewidth}{0pt}

% Adjust margins
\addtolength{\oddsidemargin}{-0.5in}
\addtolength{\evensidemargin}{-0.5in}
\addtolength{\textwidth}{1in}
\addtolength{\topmargin}{-.5in}
\addtolength{\textheight}{1.0in}

\urlstyle{same}

\raggedbottom
\raggedright
\setlength{\tabcolsep}{0in}

% Sections formatting
\titleformat{\section}{
  \vspace{-4pt}\scshape\raggedright\large
}{}{0em}{}[\color{black}\titlerule \vspace{-5pt}]

% Ensure that generate pdf is machine readable/ATS parsable
\pdfgentounicode=1

%-------------------------
% Custom commands
\newcommand{\resumeItem}[1]{
  \item\small{
    {#1 \vspace{-2pt}}
  }
}

\newcommand{\resumeSubheading}[4]{
  \vspace{-2pt}\item
    \begin{tabular*}{0.97\textwidth}[t]{l@{\extracolsep{\fill}}r}
      \textbf{#1} & #2 \\
      \textit{\small#3} & \textit{\small #4} \\
    \end{tabular*}\vspace{-7pt}
}

\newcommand{\resumeSubSubheading}[2]{
    \item
    \begin{tabular*}{0.97\textwidth}{l@{\extracolsep{\fill}}r}
      \textit{\small#1} & \textit{\small #2} \\
    \end{tabular*}\vspace{-7pt}
}

\newcommand{\resumeProjectHeading}[2]{
    \item
    \begin{tabular*}{0.97\textwidth}{l@{\extracolsep{\fill}}r}
      \small#1 & #2 \\
    \end{tabular*}\vspace{-7pt}
}

\newcommand{\resumeSubItem}[1]{\resumeItem{#1}\vspace{-4pt}}

\renewcommand\labelitemii{$\vcenter{\hbox{\tiny$\bullet$}}$}

\newcommand{\resumeSubHeadingListStart}{\begin{itemize}[leftmargin=0.15in, label={}]}
\newcommand{\resumeSubHeadingListEnd}{\end{itemize}}
\newcommand{\resumeItemListStart}{\begin{itemize}}
\newcommand{\resumeItemListEnd}{\end{itemize}\vspace{-5pt}}

%-------------------------------------------
%%%%%%  RESUME STARTS HERE  %%%%%%%%%%%%%%%%%%%%%%%%%%%%


\begin{document}

%----------HEADING----------
\begin{center}
    \textbf{\Huge \scshape Ricky Fan} \\ \vspace{1pt}
    \small 647-901-9951 $|$ \href{mailto:x@x.com}{\underline{fanh11@mcmaster.ca}} $|$ 
    \href{https://linkedin.com/in/rfan}{\underline{linkedin.com/in/rfan}} $|$
    \href{https://github.com/R-Fan9}{\underline{github.com/R-Fan9}}
\end{center}


%-----------EDUCATION-----------
\section{Education}
  \resumeSubHeadingListStart
    \resumeSubheading
      {McMaster University}{Hamilton, ON}
      {Bachelor of Applied Science in Computer Science}{Sept. 2019 -- Present}
  \resumeSubHeadingListEnd

%-----------EXPERIENCE-----------   
\section{Experience}
  \resumeSubHeadingListStart

    \resumeSubheading
      {Software Developer Intern}{Sept 2022 -- April 2023, Sept 2023 -- Present}
      {GEOTAB}{Oakville, ON}
      \resumeItemListStart
        \resumeItem{Debugged and fixed parallel tests failure due to race condition with mutex lock in C\#, reduced test failure by 98\%}
        \resumeItem{Refactored Protobuf message procecessor in Gateway to improve maintainability and testability, wrote unit tests using mocks/stubs to achieve 100\% test coverage}
        \resumeItem{Developed and maintained an automated pipeline tool to deploy test rigs by leveraging GitLab CI, Terraform, Kubernetes, Helm, and GKE}
        \resumeItem{Developed a CLI application (GatewayPiper) using C\# to deploy Terraform modules, perform health check and generate test report on Gateway test servers}
        \resumeItem{Created E2E tests for bidirectional passthrough message delivery between MyGeotab clients and devices}
        \resumeItem{Developed a C\# code documentation generator that parses StoreForward API code into Markdown syntax, streamlining Gateway's documentation process.}
        \resumeItem{Implemented a mechanism to prevent multilog preloading for inactive clients by tracking the last updated information of the client (vehicle) data offset.}
        \resumeItem{Updated Gateway resend vehicle data APIs to promptly deliver resent data at the top of the stream to support sequential data processing}
    \resumeItemListEnd
    
    \resumeSubheading
      {Software Developer Intern}{May 2023 -- Aug 2023}
      {BlackBerry QNX}{Ottawa, ON}
      \resumeItemListStart
        \resumeItem{Learned about safety certified, POSIX compliant, microkernel RTOS – QNX. Experienced with bootloader, interrupts, physical \& virtual memory and thread implementation on freestanding x86 architecture}
        \resumeItem{Developed a QNX internal PyPi package by leveraging JIRA and Jenkins APIs to automate change log request process. 
        Further optimized the module's runtime by utilizing multithreading techniques to enable multiple concurrent request processes}
      \resumeItemListEnd

    \resumeSubheading
      {Software Engineer Intern}{June 2022 -- Aug 2022}
      {Publicis Sapient}{Toronto, ON}
      \resumeItemListStart
        \resumeItem{Developed an AI chatbot for a leading US telecommunication company using React, DialogFlow, and Java Spring Boot with microservice architecture}
        \resumeItem{Developed two client – side microservices to query responses from DialogFlow AI agent, and handle SMS messaging (via Twilio API) for scheduled callback requests}
        \resumeItem{Integrated MySQL database with Spring Boot to store client information and scheduled request details}
      \resumeItemListEnd

    \resumeSubheading
      {Software Developer Intern}{Jan 2022 -- April 2022}
      {CIBC}{Toronto, ON}
      \resumeItemListStart
        \resumeItem{Built an automation tool using Python (pandas) to track and validate server jobs migration from AutoSys R11.0 and MOA to AutoSys R11.36. Documented the automation tool with Module Interface Specification (MIS), utilized Model View Controller (MVC) design pattern, unit testing done with pytest}
      \resumeItemListEnd
  \resumeSubHeadingListEnd

%-----------PROJECTS-----------
\section{Projects}
    \resumeSubHeadingListStart
      \resumeProjectHeading
          {\textbf{Th3rdplace} \emph{}}{Feb 2021 -- Aug 2022}
          \resumeItemListStart
            \resumeItem{Built customised Docker images to containerize client and server applications, enabling the app to run on local, staging and production environments}
            \resumeItem{Developed a CI/CD pipeline in GitLab by integrating Cloud Build, Cloud Run and Firebase Hosting}
            \resumeItem{Configured and deploy CRON job with Cloud Function and Cloud Scheduler to enable weekly database backup}
          \resumeItemListEnd
    \resumeSubHeadingListEnd



%-----------PROGRAMMING SKILLS-----------
\section{Technical Skills}
 \begin{itemize}[leftmargin=0.15in, label={}]
    \small{\item{
     \textbf{Languages}{: C\#, Java, Python, C/C++, Bash, JavaScript, Haskell, HTML} \\
     \textbf{Database/Frameworks/Tools}{: Firebase, MySQL, Bootstrap, Node.js, React, Redux, .Net, Django, Git, CSS, Flutter, pandas, Docker, GCP, Spring Boot, GitLab CI, Terraform} \\
    }}
 \end{itemize}


%-------------------------------------------
\end{document}